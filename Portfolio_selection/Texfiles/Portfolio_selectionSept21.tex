%\input{tcilatex}


\documentclass[ aip,jmp,reprint]{revtex4-2}
%%%%%%%%%%%%%%%%%%%%%%%%%%%%%%%%%%%%%%%%%%%%%%%%%%%%%%%%%%%%%%%%%%%%%%%%%%%%%%%%%%%%%%%%%%%%%%%%%%%%%%%%%%%%%%%%%%%%%%%%%%%%%%%%%%%%%%%%%%%%%%%%%%%%%%%%%%%%%%%%%%%%%%%%%%%%%%%%%%%%%%%%%%%%%%%%%%%%%%%%%%%%%%%%%%%%%%%%%%%%%%%%%%%%%%%%%%%%%%%%%%%%%%%%%%%%
%\usepackage{eurosym}
\usepackage{graphicx}
\usepackage{dcolumn}
\usepackage{bm}
\usepackage{hyperref}
\usepackage{caption}
\usepackage{subcaption}
\usepackage{amsmath}
\usepackage{amsfonts}
\usepackage{amssymb}

\setcounter{MaxMatrixCols}{10}
%TCIDATA{OutputFilter=LATEX.DLL}
%TCIDATA{Version=5.50.0.2960}
%TCIDATA{<META NAME="SaveForMode" CONTENT="1">}
%TCIDATA{BibliographyScheme=Manual}
%TCIDATA{LastRevised=Monday, September 21, 2020 20:19:55}
%TCIDATA{<META NAME="GraphicsSave" CONTENT="32">}
%TCIDATA{Language=American English}

%\input{tcilatex}
\begin{document}

\title{Quantum risk and portfolio management in \textbf{a }quantum mechanics
framework.}
\author{H. Khaksar$^1$}
\email{hossein.khaksar@hotmail.com}
\author{S. Nasiri$^1$}
\email{nasiri@iasbs.ac.ir}
\author{E. Haven$^2$}
\email{ehaven@mun.ca}
\author{GR. Jafari$^{1,3}$}
\email{gjafari@gmail.com}

\begin{abstract}
Owing to the globalization of\textbf{\ }the\textbf{\ }economy, the concept
of entangled markets started to form, and this occurrence has smoothed the
entrance of quantum mechanics into behavioral finance. In this manuscript,
we introduce quantum risk and perform an\textbf{\ }analy\textbf{sis} on
portfolio optimization by controlling the\textbf{\ }quantum potential. We
apply this method to eight major indices and construct a portfolio with a
minimum quantum risk. The results show quantum risk has a power law behavior
with a time-scale just as a standard deviation with different exponents.
\end{abstract}

\keywords{Finance, Portfolio and Risk management, Quantum Physics}
\maketitle

\address{$1)$ Department of Physics, Shahid Beheshti University, G.C., Evin, Tehran 19839, Iran\\ 
	$2)$ Faculty of Business Administration, Memorial University, St. John's, Canada and IQSCS, UK\\ 
	$3)$ Department of Network and Data Science, Central European University }

\section{Introduction}

The growth of the\textbf{\ }global economy has made home bound investors to
become, almost by default\textbf{, }international investors. Clearly,
portfolio theory plays a key role in international investments. Much
research has focussed on finding the optimal strategy for allocating wealth
among various assets in such a manner to reduce risk rather than maximize
returns. If we adopt a hedging perspective, we can claim this is the main
objective of portfolio theory \cite{Jorion}. Harry Markowitz's mean-variance
model \cite{Mark1}\cite{Mark2} has become a gold standard in portfolio
theory. The mean-variance portfolio optimization model is highly dependent
on the estimation errors of sample moments and includes negative weights for
large portfolios, which requires investors to take on short positions. In
the case where short positions are prohibited, constraints have to be
applied on portfolio weights in the optimization process \cite{Best}\cite%
{Green}\cite{Jagannathan}. A substantial amount of research has been
performed so as to develop a reduced error Markowitz model. In \cite{Ledoit}%
\cite{Laloux}\cite{Plerou}\cite{Rosenow}\cite{Potters}\cite{Anderberg} we
observe that the various authors used different covariant matrix estimators
in their models in order to achieve more accurate and diversified portfolios
with a lower proportion of negative weights, especially for short time
horizons (see \cite{Kondor}\cite{Pafka}). Some researchers tried non-equal
weighted historical data to distinguish between normal and more risky days
in their portfolios \cite{Coelho}. Many researchers have tried to solve the
problem by combining the investment horizon with return and risk. Bolgorian
et al.\cite{Bolgorian} found a method to introduce a portfolio with
minimized waiting time, for a particular return and known risk. In all of
these methods, variance plays an important role as a classical correlation
function in the process of optimization.

There is a widely held consensus in the academic community that the
historical return probability density function (PDF) is in general
non-Gaussian and therefore higher moments can be informative. Employing
higher moments of a PDF into the estimation of risk would require hard work
and in some cases might not even be possible.

However, one can attempt to change perspectives and try to apply a
non-classical approach through finding the optimum solution for the
portfolio problem. Although on prima facie, it may appear to be far-fetched,
but the formalism of quantum mechanics can be a perfect candidate for such a
situation, where the PDF is taken as an input of the theory and it gets rid
of all the classical problems, including moments. It was the pioneering work
of Andrei Khrennikov \cite{AKhrennikov1}\cite{AKhrennikov2} which
established the quantum mechanical approach in finance. Through the works of
Segal \cite{Segal} and Bagarello \cite{FBagarello1}\cite{FBagarello2} and
Haven et al. \cite{Haven}, amongst others, the usefulness of quantum
mechanics in their applications to finance was better understood. It was
Choustova \cite{Choustova1} who first proposed to further analyze financial
behavior using Bohmian quantum mechanics. Tahmasebi \cite{Tahmasebi} and
Shen et al.\cite{Shen} used Choustova's idea to show that historical
(public) information of an asset could be stored in a quantum potential
governing that asset (within a particular period of time). Nasiri et al.\cite%
{Nasiri1}\cite{Nasiri2} used empirical methods also proposed by Tahmasebi
and Shen et al. \cite{Shen} to analyze the role of trading volume in the
quantum potential.

In the next section, we introduce our model and formulate the questions
which we shall attempt to answer in section three. There we use a genetic
algorithm to optimize the introduced model in order to find the portfolio
and the appropriate weights for the minimum risk. In section four we compare
two types of risk and we conclude in section five.

\section{The quantum potential and a first look at what we call `quantum
risk' }

The concept of `quantum potential' is well known as being a core part of the
edifice of Bohmian mechanics \cite{Holland} which is also known as the
semi-classical approach to quantum mechanics. In this approach, the quantum
potential plays a key role in guiding the particle to its possible
trajectories. Of course, no unique particle trajectory exists in quantum
mechanics. Introducing the concept of quantum potential into an
interdisciplinary context can lead to ambiguity and requires an innovative
interpretation. The potential is easily derived from the Schr\"{o}dinger PDE
through the substitution of the wave function $\psi $ in its polar form, $%
Re^{iS}$, and resolving the equation. After the separation of real and
imaginary parts, the real part equation will be derived as Eq(\ref{eq:1}),
where the last term in Eq(\ref{eq:1}) is defined as the quantum potential
(i.e. $Q(q)$) 
\begin{equation}
\frac{\partial S}{\partial t}+(\frac{\partial S}{\partial q})^{2}+U(q)-\frac{%
\hbar ^{2}}{2mR}\frac{\partial ^{2}R}{\partial q^{2}}=0,\ \ \ \ \ \ \ \ \ \
\ \ \ Q(q)=-\frac{\hbar ^{2}}{2mR}\frac{\partial ^{2}R}{\partial q^{2}}.
\label{eq:1}
\end{equation}%
\newline
Tahmasebi \cite{Tahmasebi} and Shen \cite{Shen} showed that there exist
quantum potential walls (as well as real potential walls) for an arbitrary
(commodity) price return history. Nasiri \cite{Nasiri1} and Shen \cite{Shen}
showed that as market risk increases, the distance between the potential
walls also increases.

In this paper, we take the width of the potential walls as an effective
measure for introducing the concept of risk. In order to reach a better
understanding, we have illustrated the process in Fig(\ref{fig:1}), in which
Fig(\ref{fig:sub2}) depicts the quantum potential governing an arbitrary
observed time-series in a particular period of time shown in Fig(\ref%
{fig:sub1}). The width of the walls shown in Fig(\ref{fig:sub2}) provides
for our innovative notion which we term `quantum risk' in this particular
approach and we use this specific notion throughout the paper. We juxtapose
standard and quantum risk in the before last section of this paper. \newline

\begin{figure}[tbh]
\centering
\begin{subfigure}{0.5\textwidth}
		\centering
		\includegraphics[width=60mm]{fig0a.png}
		\caption{}
		\label{fig:sub1}
	\end{subfigure} 
\begin{subfigure}{0.5\textwidth}
		\centering
		\includegraphics[width=60mm]{fig0_b.png}
		\caption{ }
		\label{fig:sub2}
	\end{subfigure}
\caption{A schematic process of quantum potential and `quantum risk', where
a) Log-return time-series plotted for S\&P 500 and b) quantum potential
corresponding to S\&P 500 time-series.}
\label{fig:1}
\end{figure}
\newpage

\section{Portfolio optimization}

Whilst constructing a desired portfolio, it is reasonable to question why
one portfolio may be preferred over another. In this paper, we use the
notion of quantum risk, which we heuristically introduced in Fig(\ref%
{fig:sub2}), in order to optimize the quantum risk associated to a
portfolio. This portfolio is constructed by appropriate company shares. The
return of the portfolio, or index, is defined in a straightforward way as Eq(%
\ref{eq:3}): \newline
\begin{equation}
\bar{r}(t)=\sum_{i=1}^{N}\omega _{i}r_{i}(t),  \label{eq:3}
\end{equation}%
\newline
where $\omega _{i}$ is the weight and $r_{i}(t)$ is the log-return of the $%
i^{th}$ security at time $t$. One can easily construct the quantum potential 
$\frac{-\hbar ^{2}}{2mR}\frac{\partial ^{2}R}{\partial q^{2}}$ where $R(\bar{%
r})$ is the probability density function of the portfolio return index. The
measurable risk to be minimized is given by the width of the potential's
walls.

Our proposed method consists in finding a suitable choice of $\omega
_{i}^{{}}$'s with the help of a genetic algorithm in order to minimize the
risk. A graphical illustration of the process is shown in Fig(\ref{fig:2})
in which an arbitrary set of weights is considered to construct the
portfolio index. Each of these signals has been specified with their
appropriate quantum potential and their wall width, as our new `measure' of
quantum risk.\newline
\begin{figure}[tbh]
\centering
\includegraphics[width=120mm]{new1_fig0.png}
\caption{Schematic portfolio selection with quantum potential.}
\label{fig:2}
\end{figure}
\newpage In what follows, we demonstrate some simulation studies and
illustrate the optimum values for the weights of the indices composing the
appropriate portfolio. A very noticeable fact about an optimal portfolio is
that it reduces the risk of losing money. But not all the feasible
combinations of securities promise to do so. By the method introduced in the
previous section, we are going to apply portfolio management to the top
major market indices (i.e. the Dow Jones industrial, S\&P 500 composite,
FTSE 100, TOPIX , DAX 30 performance, NIKKEI 225, Korea SE composite and the
Shanghai SE A Share). The scaled quantum risk of these indices is shown in
Fig(\ref{fig3:sub2}) compared with the quantum risk of one arbitrarily
optimized portfolio. Some of the combinations (of indices) will show the
lower risk amongst all portfolios.

In this paper, we have used the genetic algorithm to find the suitable
combinations which minimize the quantum risk. It is quite clear that the
desired condition is not satisfied with only one solution. However, multiple
answers may give rise to the minimum quantum risk. Five different
combinations of indices (forming 5 different portfolios) with minimum
desired quantum risk for each portfolio is shown in fig. 3. We note we can
expect such levels of low risk given we work with portfolios of indices.  
\begin{figure}[tbh]
\centering
\includegraphics[width=100mm]{fig6pr.png}
\caption{Portfolio optimization process and selected optimized portfolios.}
\label{fig:3}
\end{figure}
\newline
\newpage 

\section{Comparing standard and quantum risk}

From an arbitrary chosen optimized portfolio, we construct scaled returns
for a $\tau $ time-scale. We claim that there exists a power-law relation
between the $Risk$, and $\tau $ as follows: 
\begin{equation}
Risk(\tau )\propto \tau ^{\alpha },
\end{equation}%
where $Risk(\tau )$ is the risk of the scaled log-return of the original
series for $\tau $ days. One can examine the $\alpha $ exponent for
different portfolios and make a comparison between their exponents. In Fig(%
\ref{fig3:sub2}), we have illustrated the amount of $\alpha $ for one of the
selected portfolios in Fig(\ref{fig:3}), where quantum risk and standard
risk plays a role for $Risk$ . We make two observations: i) the standard
risk and quantum risk have the same trend (i.e. upward); ii) an $\alpha $
close to 0.5 indicates random walk.  
\begin{figure}[tbh]
\begin{subfigure}{0.49\textwidth}
		\centering
		\includegraphics[width=70mm]{fig1.png}
		\caption{a}
		\label{fig3:sub1}
	\end{subfigure}
\begin{subfigure}{0.49\textwidth}
		\centering
		\includegraphics[width=70mm]{fig3_a.png}
		\caption{b}
		\label{fig3:sub2}
	\end{subfigure}
\caption{Log-return time-series and its appropriate quantum potential risk
plotted for S\&P 500 index.}
\label{fig:4}
\end{figure}

In Fig(\ref{fig3:sub2}), we have demonstrated the scaled risk for different
portfolios, in order that one can get a better feeling towards comparing the
risk of each individual index with its standard deviation and also the
optimized portfolio, for a determined period of time (from Dec. 1994 to Dec.
2019). Fig. 4a shows the normalized quantum risk of major indices introduced
in the previous section, compared with their normalized standard notion for
risk which is standard deviation. As one can follow from Fig. 4a, the Dow
Jones and S\&P 500 have the lowest risk amongst all the indices. The
Shanghai index itself got the highest risk among these 8 indices. The top
bars in Fig. 4a show the risk and standard deviation of one selected
optimized portfolio. It is clear that both the quantum risk and standard
deviation of the selected portfolio is less than all the indices composing
the portfolio.

\newpage

\section{Conclusion}

In this paper, we showed how a simple concept of the quantum mechanical
formalism can begin to aid us in risk management. We introduced a method
which, if we control the quantum potential, results in extracting a specific
measure of risk information of an index. Since the quantum potential has
been shown to help in analyzing the coupling between markets, the bridge
between market risk and systematic risk may be worthy of further
consideration in future work.

\newpage

\section{References}

\begin{thebibliography}{99}
\bibitem{Jorion} Jorion, P. International portfolio diversification with
estimation risk. \textit{J. Bus.} \textbf{1985}, 259-278.

\bibitem{Mark1} Markowitz, H. M. Foundations of portfolio theory. \textit{J.
Financ. }\textbf{1991, }46 (2), 469-477.

\bibitem{Mark2} Markowitz, H. M. The early history of portfolio theory:
1600-1960. \textit{Financ. Anal. J. }\textbf{1999}, 55 (4), 5-16.

\bibitem{Best} Best, M.J.; Grauer, R.G. \ Positively weighted
minimum-variance portfolios and the structure of asset expected returns. 
\textit{J. Financ. Quant. Anal.}  \textbf{1992, }27 (4), 513-537.

\bibitem{Green} Green, R. C.; Hollifield, B. When will mean-variance
efficient portfolios be well diversified? \textit{J. Financ. }\textbf{1992 }%
7 (5), 1785-1809.

\bibitem{Jagannathan} Jagannathan, R.; Ma, T. Risk reduction in large
portfolios: why imposing the wrong constraints helps. \textit{J. Financ. }%
\textbf{2003}, 58 (4), 1651-1683.

\bibitem{Ledoit} Ledoit, O.; Wolf, M.  Improved estimation of the covariance
matrix of stock returns with an application to portfolio selection. \textit{%
J. Empir. Financ. }\textbf{2003, }10(5), 603-621.

\bibitem{Laloux} Laloux, L. et al. Noise dressing of financial correlation
matrices. \textit{Phys. Rev. Lett. }\textbf{1999}, 83 (7), 1467.

\bibitem{Plerou} Plerou, V. et al. \ Universal and nonuniversal properties
of cross correlations in financial time series. \textit{Phys. Rev. Lett. }%
\textbf{1999, }83 (7), 1471.

\bibitem{Rosenow} Rosenow, B. et al. Portfolio optimization and the random
magnet problem. \textit{Europhys. Lett. }\textbf{2002}, 59(4), 500.

\bibitem{Potters} Potters, M.; Bouchaud, J.P.; Laloux, L.  Financial
applications of random matrix theory: old laces and new pieces. arXiv
preprint physics/0507111 \textbf{2005}.

\bibitem{Anderberg} Anderberg, M. R. \textit{Cluster Analysis for
Applications}. A Volume in Probability and Mathematical Statistics: a Series
of Monographs and Textbooks. Academic Press, 1973.

\bibitem{Kondor} Imre, K.; Szilard, P.; Gabor, N. Noise sensitivity of
portfolio selection under various risk measures. \textit{J. Bank. Financ. }%
\textbf{2007, } 31(5), 1545-1573.

\bibitem{Pafka} Szilard, P.; \ Imre K. Noisy covariance matrices and
portfolio optimization II. \textit{Physica A} \textbf{2003, }319, 487-494.

\bibitem{Coelho} Coelho, R. et al. The evolution of interdependence in world
equity markets: evidence from minimum spanning trees. \textit{Physica A} 
\textbf{2007, }376, 455-466.

\bibitem{Bolgorian} Meysan, B.; Shirazi, A.H.;  Jafari G.R. Portfolio
selection using level crossing analysis. \textit{Int. J. Mod. Phys.} \textbf{%
2011, }22 (08), 841-848.

\bibitem{AKhrennikov1} Khrennikov, A. Classical and quantum mechanics on
information spaces with applications to cognitive, psychological, social,
and anomalous phenomena. \textit{Found. Phys.} \textbf{1999, }29 (7),
1065-1098.

\bibitem{AKhrennikov2} Khrennikov, A. \textit{Ubiquitous quantum structure:
from psychology to finance.} Springer Nature, 2010.

\bibitem{Segal} Segal, W.; Segal, I.E. \ The Black-Scholes pricing formula
in the quantum context. \textit{P. Natl. Acad. Sci. USA} \textbf{1998},
95(7), 4072-4075.

\bibitem{FBagarello1} Bagarello, F. \textit{Quantum dynamics for classical
systems.} J. Wiley, Hoboken-N.J., 2013.

\bibitem{FBagarello2} Bagarello, F. An operatorial approach to stock
markets. \textit{J. Phys. A} 39, 6823.

\bibitem{Haven} Haven, E.; Khrennikov, A.; Ma, C.; Sozzo, S. Special section
on \textquotedblleft Quantum Probability Theory and its Economic
Applications\textquotedblright . \textit{J. Math. Econ. }\textbf{2018}, 78,
127-197.

\bibitem{Choustova1} Choustova, O. \ Quantum probability and financial
market. \textit{Inform. Sciences }\textbf{2009}, 179(5), 478-484.

\bibitem{Tahmasebi} Tahmasebi F.; Meskinimood, S.; Namaki, A.; Farahani,
S.V.; Jalalzadeh, S.; Jafari, G. Financial market images: a practical
approach owing to the secret quantum potential. \textit{Eur. Phys. Lett.} 
\textbf{2015, }109 (3), 30001.

\bibitem{Shen} Shen C.; Haven, E. Using empirical data to estimate potential
functions in commodity markets: some initial results. \textit{Int. J. Theor.
Phys. } \textbf{2017, }56 (12), 4092--4104.

\bibitem{Nasiri1} Nasiri, S.; Bektas, E.; Jafari, G. \ Risk information of
stock market using quantum potential constraints. In: \textit{Emerging
Trends in Banking and Finance}. Springer. 2018. pp. 132-138.

\bibitem{Nasiri2} Nasiri, S. ; Bektas, E.; Jafari, G. The impact of trading
volume on the stock market credibility: Bohmian quantum potential approach. 
\textit{Physica A} \textbf{2018, }512, 1104-1112.

\bibitem{Holland} Holland, P. \textit{The Quantum Theory of Motion: An
Account of the de Broglie-Bohm Causal Interpretation of Quantum Mechanics.}
Cambridge University Press, 1995. 
\end{thebibliography}

\end{document}
