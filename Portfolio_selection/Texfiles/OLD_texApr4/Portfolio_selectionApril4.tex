%\input{tcilatex}


\documentclass[ aip,jmp,reprint]{revtex4-2}
%%%%%%%%%%%%%%%%%%%%%%%%%%%%%%%%%%%%%%%%%%%%%%%%%%%%%%%%%%%%%%%%%%%%%%%%%%%%%%%%%%%%%%%%%%%%%%%%%%%%%%%%%%%%%%%%%%%%%%%%%%%%%%%%%%%%%%%%%%%%%%%%%%%%%%%%%%%%%%%%%%%%%%%%%%%%%%%%%%%%%%%%%%%%%%%%%%%%%%%%%%%%%%%%%%%%%%%%%%%%%%%%%%%%%%%%%%%%%%%%%%%%%%%%%%%%
\usepackage{eurosym}
\usepackage{graphicx}
\usepackage{dcolumn}
\usepackage{bm}
\usepackage{hyperref}
\usepackage{caption}
\usepackage{subcaption}
\usepackage{amsmath}
\usepackage{amsfonts}
\usepackage{amssymb}

\setcounter{MaxMatrixCols}{10}
%TCIDATA{OutputFilter=LATEX.DLL}
%TCIDATA{Version=5.50.0.2960}
%TCIDATA{<META NAME="SaveForMode" CONTENT="1">}
%TCIDATA{BibliographyScheme=Manual}
%TCIDATA{LastRevised=Tuesday, April 07, 2020 12:12:47}
%TCIDATA{<META NAME="GraphicsSave" CONTENT="32">}

\input{tcilatex}
\begin{document}

\title{Quantum risk and portfolio management in \textbf{a }quantum mechanics
framework.}
\author{H. Khaksar$^1$}
\email{hossein.khaksar@hotmail.com}
\author{S. Nasiri$^1$}
\email{nasiri@iasbs.ac.ir}
\author{E. Haven$^2$}
\email{ehaven@mun.ca}
\author{GR. Jafari$^{1,3}$}
\email{gjafari@gmail.com}

\begin{abstract}
Owing to the globalization of the economy, the concept of entangled
markets started to form, which has smoothed the entrance of quantum
mechanics into behavioral finance. In this manuscript, we introduce
quantum risk and perform an analysis on portfolio
optimization by controlling the quantum potential. We apply this
method to eight major indices and construct a portfolio with a minimum
quantum risk. The results show quantum risk has a power law behavior with a
time-scale just as a standard deviation with different exponents.
\end{abstract}

\keywords{Finance, Portfolio and Risk management, Quantum Physics}
\maketitle

\address{$1)$ Department of Physics, Shahid Beheshti University, G.C., Evin, Tehran 19839, Iran\\ 
	$2)$ Faculty of Business Administration, Memorial University, St. John's, Canada and IQSCS, UK\\ 
	$3)$ Department of Network and Data Science, Central European University }

\section{Introduction}

\textbf{Thank you for allowing me to study the paper!}

It was Warren Buffet who said that in years from now, cash will be
worth less. The growth of the global economy, has made home
bound investors to become, almost by default, international investors. Much
research has focussed on finding the optimal strategy for allocating wealth
among various assets in such a manner to reduce risk rather than maximize
returns (which is the main objective of portfolio theory)\cite{Jorion}. It
was Harry Markowitz's mean-variance model\cite{Mark1,Mark2} which
mobilized portfolio theory in the early 1950's. The mean-variance portfolio
optimization model was highly dependent on the estimation errors of sample
moments and included negative weights for large portfolios, which required
investors to take on short positions. In the case where short positions were
prohibited, constraints had to be applied on portfolio weights in the
optimization process\cite{Best,Green,Jagannathan}. Enormous
amounts of research have been done to develop Markowitz's model in order to
reduce its errors. Ledoit, Laloux among others \cite{Ledoit,Laloux,Plerou,Rosenow,Potters,Anderberg} used different covariant matrix
estimators in their models to get more accurate and diversified portfolios
than Markowitz's, and with a lower proportion of negative weights,
especially for short time horizons \cite{Kondor,Pafka,Kondor2}. Coelho tried non-equal weighted historical data to distinguish
between normal and more risky days in their portfolios\cite{Coelho}. Many
researchers have tried to solve the problem by combining the investment
horizon with return and risk. Bolgorian et al.\cite{Bolgorian} found a
method to introduce a portfolio with minimized waiting time, for a
particular return and known risk. In all of these methods, variance plays an
important role as a classical correlation function in the process of
optimization.

There is a widely held consensus in the academic community that the
historical return probability density function (PDF) is in general
non-Gaussian and therefore higher moments are informative. Employing higher
moments of a PDF into the estimation of risk would require hard work and in
some cases might not even be possible.

However, one can attempt to change perspectives and try to apply a
non-classical approach through finding the optimum solution for the
portfolio problem. Although on prima facie, it may appear to be far-fetched,
but the formalism of quantum mechanics can be a perfect candidate for such a
situation, where the PDF is taken as an input of the theory and it gets rid
of all the classical problems, including moments. It was the pioneering work
of Andrei Khrennikov\cite{Khrennivov1} which established the quantum
mechanical approach in finance. Through the works of Segal\cite{Segal} and
Haven\cite{Haven2}, amongst others, the usefulness of quantum mechanics in
their applications to finance was better understood. It was Choustova\cite%
{Choustova1} who first proposed to further analyze financial behavior using
Bohmian quantum mechanics. Tahmasebi\cite{Tahmasebi} and Shen et al.\cite%
{Shen} used Choustova's idea to show that historical information of an asset
could be stored in a quantum potential governing that asset (within a
particular period of time). Nasiri et al.\cite{Nasiri1,Nasiri2} used
empirical methods proposed by Tahmasebi and Shen to analyze the role of
trading volume in the quantum potential.

In the next section, we introduce our model and formulate the questions
which we shall attempt to answer in section three. There we use a genetic
algorithm to optimize the introduced model in order to find the portfolio
and the appropriate weights for the minimum risk. In section four we compare
two typoes of risk and we conclude in section five.

\section{Quantum potential to `quantum' risk}

The concept of quantum potential is well known as being part of the edifice
of Bohmian mechanics [reference needed] which is also known as the
semi-classical approach to quantum mechanics. In this approach, the quantum
potential plays a key role in guiding the particle its possible
trajectories. Of course no unique particle trajectory exists in quantum
mechanics. Introducing the concept of quantum potential into an
interdisciplinary context can be ambiguous and may require an innovative
interpretation. The potential is easily derived from Schr\"{o}dinger
equation through the substitution of the wave function $\psi $ with its
polar form $Re^{iS}$ and resolving the equation. After the separation of
real and imaginary parts of the Schr\"{o}dinger equation, the real part
equation will be derived as Eq(\ref{eq:1}), where the last term in Eq(\ref%
{eq:1}) is defined as quantum potential defined in Eq(\ref{eq:1}) 
\begin{equation}
\frac{\partial S}{\partial t}+(\frac{\partial S}{\partial q})^{2}+U(q)-\frac{%
\hbar ^{2}}{2mR}\frac{\partial ^{2}R}{\partial q^{2}}=0,\ \ \ \ \ \ \ \ \ \
\ \ \ Q(q)=-\frac{\hbar ^{2}}{2mR}\frac{\partial ^{2}R}{\partial q^{2}}.
\label{eq:1}
\end{equation}%
\newline
Tahmasebi\cite{Tahmasebi} and Shen\cite{Shen} showed that there exist
quantum potential walls for an arbitrary price return history and Nasiri\cite%
{Nasiri1} showed that as market risk increases, the distance between the
potential walls also increases.

In this paper, we take the width of the potential walls as an effective
measure for introducing the concept of risk. In order to reach a better
understanding, we have illustrated the process in Fig(\ref{fig:1}), in which
Fig(\ref{fig:sub2}) depicts the quantum potential governing an arbitrary
observed time-series in a particular period of time shown in Fig(\ref%
{fig:sub1}). The width of the walls shown in Fig(\ref{fig:sub2}) provides
for our innovative notion which we term `quantum risk' in this particular
approach and we use this specific notion throughout the paper.\newline
\textbf{[question: width of walls: but width of walls on real potentials
coincide...the idea of quantum risk is interesting - but i think it relates
not to the width of the walls of the potential well - since wall width for
both real potential and quantum coincide. It may be a idea to look at force
- either derived from real or quantum potential - and the gradient of this
force. I also think that quantum risk as defined here - is simply measuring
maximum spread level of return? ]}

\begin{figure}[htb]
\centering
\begin{subfigure}{0.5\textwidth}
		\centering
		\includegraphics[width=60mm]{fig0a.png}
		\caption{}
		\label{fig:sub1}
	\end{subfigure} 
\begin{subfigure}{0.5\textwidth}
		\centering
		\includegraphics[width=60mm]{fig0_b.png}
		\caption{ }
		\label{fig:sub2}
	\end{subfigure}
\caption{An schematic process of quantum potential and quantum risk, where
a) Log-return time-series plotted for S\&P 500 and b) quantum potential
corresponding to S\&P 500 time-series.}
\label{fig:1}
\end{figure}
\newpage

\section{Portfolio optimization}

Whilst constructing a desired portfolio, it is reasonable to question why
one portfolio may be preferred over another. In this paper, we use the
notion of quantum risk introduced in Fig(\ref{fig:sub2}), in order to
optimize the quantum risk associated to a portfolio, which is constructed by
appropriate company shares. The return of the portfolio, or index is defined
in a straightforward way as Eq(\ref{eq:3}): \newline
\begin{equation}
\bar{r}(t)=\sum_{i=1}^{N}\omega _{i}r_{i}(t),  \label{eq:3}
\end{equation}%
\newline
where $\omega _{i}$ is the weight and $r_{i}(t)$ is the log-return of the $%
i^{th}$ security in time $t$. One can easily construct the quantum potential 
$\frac{-\hbar ^{2}}{2mR}\frac{\partial ^{2}R}{\partial q^{2}}$ where $R(\bar{%
r})$ is the probability density function  of the portfolio return index. The
measurable risk to be minimized is the width of the potential's walls.
[again real and quantum walls coincide...]

Our method is to find a suitable choice of $\omega _{i}^{{}}$'s with the
help of a genetic algorithm in order to minimize the risk. A graphical
illustration of the process is shown in Fig(\ref{fig:2}) in which an
arbitrary set of weights is considered to construct the portfolio index.
Each of these signals has been specified with their appropriate quantum
potential and their wall width as our new notion of quantum risk. \textbf{%
[Question: weights in the figure should w1, w2,....wn - not w3]}\newline
\begin{figure}[tbh]
\centering
\includegraphics[width=120mm]{new1_fig0.png}
\caption{Schematic portfolio selection with quantum potential.}
\label{fig:2}
\end{figure}
\newpage In the following we demonstrate some case studies and illustrate
the optimum values for the weights of the shares composing the appropriate
portfolio. A very noticeable fact about an optimal portfolio is that it
reduces the risk of loosing money. But not all the feasible combinations of
securities promise us to do so. By the method introduced in the previous
section, we are going to apply portfolio management to the top major market
indices, namely Dow Jones industrial, S\&P 500 composite, FTSE 100, TOPIX ,
DAX 30 performance, NIKKEI 225, Korea SE composite and the Shanghai SE A
Share. The scaled quantum risk of these indices has been shown in the Fig(%
\ref{fig3:sub2}) compared with the quantum risk of one arbitrary optimized
portfolio. Some of the combinations will show the lower risk among all. In
this work we have tried the genetic algorithm to find the suitable
combinations which minimizes the quantum risk.[  It is pretty obvious that
the desired condition is not satisfied with only one solution however lots
of answers may give rise to the minimum quantum risk. Five different
combinations of indices (forming 5 different portfolios) with minimum
desired quantum risk for its portfolio is shown in the Fig(\ref{fig:2}).
[figure 3!!] [Question: \textbf{The issue may be here that the return is not
juxtaposed next to the risk level. The green portfolio may be useful to have
from a risk level but how does it compare to the other portfolios in terms
of expected return?] }
\begin{figure}[tbh]
\centering
\includegraphics[width=100mm]{fig6pr.png}
\caption{Portfolio optimization process and selected optimized portfolios.}
\label{fig:3}
\end{figure}
\newline
The degeneracy is pretty expected [\textbf{question: IT IS UNCLEAR WHAT IS
MEANT WITH DEGENERACY}] with the fact that the correlation companies [?],
especially the major ones has been risen enormously since last decades [%
\textbf{My apologies - it is a little unclear.}]. Having the degenerate
portfolios can come to help in some senses, managing the dynamics of our
portfolio in time, walking through various minimum quantum risk portfolios,
among others. \textbf{[Question: I think this could need a bit more
explanation]} ... \newpage 

\section{Comparing Standard and quantum risk}

In the previous section, we realized that the optimum risk is highly
degenerate \textbf{[Question: thanks for indicating what you\ mean?]} among
constructed portfolios. This degeneracy has allowed us to choose different
portfolios in each different distinct periods. In this section, we introduce
another useful method towards distinguishing among degenerate portfolios.
From an arbitrary chosen optimized portfolio, we construct scaled returns
for a $\tau $ time-scale. We claim that there exists a power-law relation
between the $Risk$, and $\tau $ as follows: 
\begin{equation}
Risk(\tau )\propto \tau ^{\alpha },
\end{equation}%
Where $Risk(\tau )$ is the risk of the scaled log-return of the original
series for $\tau $ days. One can examine the $\alpha $ exponent for
different portfolios and make a comparison between their exponents. In Fig(%
\ref{fig3:sub2}), we have illustrated the amount of $\alpha $ for one of the
selected portfolios in Fig(\ref{fig:3}), where quantum risk and standard
risk plays a role for $Risk$ . 
\begin{figure}[tbh]
\begin{subfigure}{0.49\textwidth}
		\centering
		\includegraphics[width=70mm]{fig1.png}
		\caption{a}
		\label{fig3:sub1}
	\end{subfigure}
\begin{subfigure}{0.49\textwidth}
		\centering
		\includegraphics[width=70mm]{fig3_a.png}
		\caption{b}
		\label{fig3:sub2}
	\end{subfigure}
\caption{Log-return time-series and its appropriate quantum potential
plotted for S\&P 500 index.}
\label{fig:4}
\end{figure}

[colours seem to be reversed in the figure - blue is quantum risk - but not
in figure b.?

In Fig(\ref{fig3:sub2}), we have demonstrated the scaled risk for different
securities, in order that one can get a better feeling towards comparing the
risk of each individual security with its standard deviation and also the
optimized portfolio, for a determined period of time(From Dec. 1994 to Dec.
2019). Fig(\ref{fig3:sub2}) shows the normalized quantum risk of major
indices introduced in the previous section, compared with their normalized
standard notion existing for risk which is standard deviation. [\textbf{%
Question: SO HERE YOU DISTINGUISH BETWEEN WALL WIDTH AND AVERAGE (MAX SD) ON
THE RETURNS. A POSSIBLE QUESTION: WHY WOULD THOSE TWO DIFFER THAT MUCH -
KNOWING THAT WALL WIDTH OF REAL AND QUANTUM POTENTIAL COINCIDE (THE PDF ON
THE REAL POTENTIAL WILL GIVE INDICATING OF SD LEVEL?)] }As one can follow
from Fig(\ref{fig3:sub2}) the Dow Jones and S\&P 500 have the lower risk of
all and Shanghai index itself got the highest risk among these 8 indices.
Above all these securities in Fig(\ref{fig3:sub2}), the risk and STD of one
selected optimized portfolio has been drawn. It is clear that both the
quantum risk and STD of the selected portfolio is less than all the
securities composing the portfolio. \ \textbf{[question: THIS IS TO BE
EXPLAINED BETTER - IT IS NOT CLEAR. BUT IT IS A VERY INTERESTING
DEVELOPMENT. ]}

\newpage

\section{Conclusion}

Risk has been the most important variable for investors for a long period of
time. In spite of the fact that lots of research has been done in this
particular context, we still welcome any innovative idea which formalizes
better risk management. Recently, quantum mechanical applications in
financial analysis attracted the attention of some researchers. In this
work, we showed how quantum mechanics can lead us to perform portfolio and
risk management by introducing a method by which, if we control the quantum
potential, results in extracting risk information of a security. Since the
quantum potential has been shown to analyze the coupling between markets,
the bridge between market risk and systematic risk can be connected in
further analysis [what do you mean?].

\newpage

\section{Refrences [there are spelling mistakes in the reference list - i
can adjust]}

\begin{thebibliography}{99}
\bibitem{Jorion} Jorion, Philippe. "International portfolio diversification
with estimation risk." Journal of Business (1985): 259-278.

\bibitem{Mark1} Markowitz, Harry M. "Foundations of portfolio theory." The
journal of finance 46.2 (1991): 469-477.

\bibitem{Mark2} Markowitz, Harry M. "The early history of portfolio theory:
1600\^{a}\euro ``1960." Financial analysts journal 55.4 (1999): 5-16.

\bibitem{Best} Best, Michael J., and Robert R. Grauer. "Positively weighted
minimum-variance portfolios and the structure of asset expected returns."
Journal of Financial and Quantitative Analysis 27.4 (1992): 513-537.

\bibitem{Green} Green, Richard C., and Burton Hollifield. "When will mean?variance efficient portfolios be well diversified?." The Journal of Finance 47.5 (1992): 1785-1809.

\bibitem{Jagannathan} Jagannathan, Ravi, and Tongshu Ma. "Risk reduction in
large portfolios: Why imposing the wrong constraints helps." The Journal of
Finance 58.4 (2003): 1651-1683.

\bibitem{Ledoit} Ledoit, Olivier, and Michael Wolf. "Improved estimation of
the covariance matrix of stock returns with an application to portfolio
selection." Journal of empirical finance 10.5 (2003): 603-621.

\bibitem{Laloux} Laloux, Laurent, et al. "Noise dressing of financial
correlation matrices." Physical review letters 83.7 (1999): 1467.

\bibitem{Plerou} Plerou, Vasiliki, et al. "Universal and nonuniversal
properties of cross correlations in financial time series." Physical review
letters 83.7 (1999): 1471.

\bibitem{Rosenow} Rosenow, Bernd, et al. "Portfolio optimization and the
random magnet problem." EPL (Europhysics Letters) 59.4 (2002): 500.

\bibitem{Potters} Potters, Marc, Jean-Philippe Bouchaud, and Laurent Laloux.
"Financial applications of random matrix theory: Old laces and new pieces."
arXiv preprint physics/0507111 (2005).

\bibitem{Anderberg} Anderberg, Michael R. Cluster analysis for applications:
probability and mathematical statistics: a series of monographs and
textbooks. Vol. 19. Academic press, 2014.

\bibitem{Kondor} Kondor, Imre, Szil\~{A}\textexclamdown rd Pafka, and G\~{A}%
\textexclamdown bor Nagy. "Noise sensitivity of portfolio selection under
various risk measures." Journal of Banking \& Finance 31.5 (2007): 1545-1573.

\bibitem{Pafka} Pafka, Szil\~{A}\textexclamdown rd, and Imre Kondor. "Noisy
covariance matrices and portfolio optimization II." Physica A: Statistical
Mechanics and its Applications 319 (2003): 487-494.

\bibitem{Kondor2} Kondor, Imre, et al. "Estimation Noise in Portfolio
Optimization with Absolute Deviation." (2004).

\bibitem{Coelho} Coelho, Ricardo, et al. "The evolution of interdependence
in world equity markets\^{a}\euro ''Evidence from minimum spanning trees."
Physica A: Statistical Mechanics and its Applications 376 (2007): 455-466.

\bibitem{Bolgorian} Bolgorian, Meysam, A. H. Shirazi, and G. R. Jafari.
"Portfolio Selection Using Level Crossing Analysis." International Journal
of Modern Physics C 22.08 (2011): 841-848.


\bibitem{Khrennivov1} Khrennivov, Andrei. "Classical and quantum mechanics
on information spaces with applications to cognitive, psychological, social,
and anomalous phenomena." Foundations of Physics 29.7 (1999): 1065-1098.

\bibitem{Segal} Segal, Wiliam, and I. E. Segal. "The Black\^{a}\euro
``Scholes pricing formula in the quantum context." Proceedings of the
National Academy of Sciences 95.7 (1998): 4072-4075.

\bibitem{Haven2} Haven, Emmanuel. "The variation of financial arbitrage via
the use of an information wave function." International Journal of
Theoretical Physics 47.1 (2008): 193-199.

\bibitem{Choustova1} Choustova, Olga. "Quantum probability and financial
market." Information Sciences 179.5 (2009): 478-484.

\bibitem{Tahmasebi} Tahmasebi F., S. Meskinimood, A. Namaki, S. V. Farahani,
S. Jalalzadeh and G. Jafari, Financial market images: a practical approach
owing to the secret quantum potential. EPL: Europhysics Letters \textbf{109
(3)}, 30001 (2015).

\bibitem{Shen} Shen C. and E. Haven, Using empirical data to estimate
potential functions in commodity markets: some initial results.
International Journal of Theoretical Physics \textbf{56 (12)}, 4092--4104
(2017).

\bibitem{Nasiri1} Nasiri, Sina, Eralp Bektas, and Gholamreza Jafari. "Risk
Information of Stock Market Using Quantum Potential Constraints." Emerging
Trends in Banking and Finance. Springer, Cham, 2018. 132-138.

\bibitem{Nasiri2} Nasiri, Sina, Eralp Bektas, and G. Reza Jafari. "The
impact of trading volume on the stock market credibility: Bohmian quantum
potential approach." Physica A: Statistical Mechanics and its Applications
512 (2018): 1104-1112.

\bibitem{Haven} Haven, Emmanuel, \ A. Khrennikov, C. Ma and S. Sozzo,
Special section on \textquotedblleft Quantum Probability Theory and its
Economic Applications\textquotedblright. Journal of Mathematical Economics 
\textbf{78}, 127-197 (2018).
\end{thebibliography}

\end{document}
